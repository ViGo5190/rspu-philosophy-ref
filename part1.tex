\chapter{История развития информатики} \label{chapt1}

Развитие вычислительной техники в XX веке стало основополагающим в  формировании информатики как науки.

Понятие информатики возникло  во Франции в 60-х гг. Так решили назвать область знаний, изучающую применение электронных вычислительных машин для автоматизации обработки информации. Слово информатика образовано путем слияния французских слов информация и автоматика. В англоязычных странах вместо «информатики» часто используют термин «computer science» (компьютерная наука).

В 40-х годах XX века ещё не существовало "информатики" - была лишь "компьютерная наука" ( хотя сейчас эти понятия и принято считать тождественными ), представляющая собой объединение возможностей электронно-вычислительных машин, математической логики и теории алгоритмов. Усовершенствования ЭВМ позволило в компьютерной науке появиться новым направлениям \cite{bib102}.


\section{Предыстория информатики} \label{sect1_1}

\textbf{Предыстория} информатики уходит корнями в историю человеческого общества. Предысторию информатики можно разбить на несколько этапов, каждый из которых характеризуется резким изменениям в возможностях передачи, хранения и обработки информации.

\subsection{Начальный этап предыстории информатики} \label{subsect1_1_1}

Освоение человеком развитой устной речи является начальным этапом предыстории информатики. Членораздельная речь и язык стали основанием для формирования специфических социальных средств хранения и передачи информации. 

\subsection{Возникновение письменности} \label{subsect1_1_2}

Вступление человечества в эпоху письменности можно связать с первыми наскальными рисунками, выполненными почти 30 тысячелетий назад. Фактически это означает, что был найден самый первый, более надежный по сравнению с человеческим мозгом, способ долговременного хранения и передачи информации. Именно тогда появились и первые инструменты для ее «записи» на носитель. К этому же времени можно отнести и начало использования пальцев рук и камешков для выполнения простейших вычислений. Итак, появились первые приспособления для хранения и обработки информации, поскольку и наскальные рисунки, и камешки для счета в принципе представляют собой информационные образы неких реальных объектов. С этого времени начинается отсчет ручного этапа в развитии средств обработки информации. Заметим, что между появлением первых орудий материального труда и первых приспособлений для долговременной фиксации информации и ее обработки прошло около миллиона лет.

За тридцать тысяч лет, прошедших со времени появления письменности, человечество сумело найти и применить на практике немало различных носителей информации – камень, кость, дерево, глину, папирус, бумагу. В то же время, несмотря на существенное усовершенствование орудий материального труда, количество новых приспособлений для обработки информации практически оставалось на одном и том же уровне. Это пальцевый счет, камешки, насечки, узелковый счет в доколумбовой Америке, абак – глиняная пластинка с желобками, в которых размещались камешки, русские счеты и некоторые другие аналогичные приспособления – это все, чем может похвастаться человечество за этот период. Тем не менее, появление письменности можно считать исторически первым этапом развития информационных технологий, которое существенно ускорило развитие человеческого общества \cite{bib103}.

\subsection{Книгопечатание} \label{subsect1_1_3}

Изобретение в середине XV века Иоганном Гутенбергом печатного станка, который позволял тиражировать знания на долговременном бумажном носителе информации, стало началом эры бумажной информатики и явилось мощнейшим катализатором промышленной революции XVIII века. Знания, тиражируемые и широко распространяемые по всему миру, оказывали значительное влияние на разработку новых устройств, технологических приемов и т. д. А внедрение новых изобретений в производственный процесс, в свою очередь, порождало новые знания, также распространяемые и тиражируемые с помощью книг. Начала раскручиваться спираль научно-технического прогресса.

По сравнению с предыдущим этапом предыстории информатики на этом этапе не столько увеличивалась возможность хранения информации (хотя и здесь был выигрыш: письменный источник – это часто один-единственный экземпляр, печатная книга – это целый тираж экземпляров, а следовательно, и малая вероятность потери информации при хранении), сколько повысилась доступность информации и точность ее воспроизведения.  \cite{bib104}

\subsection{Механическая обработка информации} \label{subsect1_1_4}


Почти через 200 лет после появления печатного станка были разработаны первые устройства для механической обработки числовой информации, наиболее простой и, с другой стороны, наиболее важной для того периода разновидности информации. С этого времени начинается механический этап в развитии средств обработки информации. Основное отличие этого этапа состоит в том, что вычисления осуществляются путем механических перемещений различных узлов – рычагов, валиков различной формы, зубчатых колес и т. д.
Первой известной попыткой построения такого механизма является относящийся примерно к 1500 году эскиз суммирующего устройства Леонардо да Винчи. К сожалению, в то время построить по этому эскизу реальное счетное устройство не удалось. А первое действующее устройство для выполнения сложения было создано только в 1623 году Вильгельмом Шиккардом. Он называл свое изобретение «суммирующими часами», так как оно было создано (к сожалению, в единичном экземпляре) на базе механических часов. Блез Паскаль в 1641-1645 годах разработал суммирующую машину, которая получила широкую известность и была выпущена целой серией в 50 машин (8 экземпляров дошло до наших дней). А Готфриду Лейбницу в 1671-1674 годах удалось построить арифмометр – машину для выполнения всех четырех арифметических операций.
Кстати, абак и счеты, если говорить более точно, не являются устройствами для обработки информации в полном смысле этого слова. Это устройства только для хранения числовой информации в процессе ее обработки, поскольку арифметические операции над числами все-таки производятся человеком. В то время как суммирующую машину Паскаля и арифмометр Лейбница уже можно считать своеобразными «механическими вычислителями», которые выполняли целый ряд вычислительных действий без вмешательства человека.

Так почти 350 лет тому назад появились предшественники современных микрокалькуляторов. Вся эта группа средств обработки информации, включающая в себя и суммирующие «часы» Шиккарда, и машину Паскаля, и широко распространенные в конце XIX и начале XX веков арифмометры Томаса и Орднера, и нынешние микрокалькуляторы, отличается тем, что человек непосредственно участвует в вычислительном процессе на всех его этапах. В частности, человек не только определяет последовательность выполняемых действий, но и осуществляет собственно вычисления.

В ходе промышленной революции появились и стали широко использоваться бумажные ленты и карты с отверстиями – перфоленты и перфокарты, которые являются разновидностью долговременных носителей информации. С помощью определенных комбинаций отверстий на перфолентах и перфокартах задавался конкретный план работы различных устройств. Примером такого рода устройств является автоматический ткацкий станок, изобретенный во Франции в 1804-1808 годах Жозефом Жаккардом. Работой этого станка управляла перфокарта с заранее нанесенными на нее отверстиями. Наличие или отсутствие отверстия в перфокарте заставляло подниматься или опускаться нить при одном ходе челнока. Станок Жаккарда был первым массовым промышленным устройством, автоматически (то есть без прямого вмешательства человека) работающим по заданному плану. План выполнения действий является особого рода информацией, использование которой позволяет достичь заданной цели.

Таким образом, в частном случае производства роль человека свелась к составлению плана выполнения нужных действий, а сами действия уже выполнялись без участия человека – автоматически. Естественным образом должна была возникнуть мысль о том, что машине можно поручить не только выполнение действий по изготовлению тканей. По-видимому, можно попытаться поручить ей и выполнение некоторых вычислений, которые, как уже было отмечено ранее, представляли собой наиболее важную в то время разновидность действий по обработке информации. Такая мысль возникла у английского математика Чарльза Бэббиджа в начале XIX века. В 1822 году он опубликовал статью с описанием так называемой «разностной» машины, предназначенной для вычисления и печати таблиц математических функций. Затем Бэббидж начал работать над проектом реализации машины, которую впоследствии стали называть «аналитической». Первый эскиз этой машины появился в 1834 году. Однако, несмотря на несколько десятилетий работы и затраченные усилия, Бэббиджу не удалось реализовать свою идею, в основном из-за несовершенства материальной и технической базы того периода. Проект машины Бэббиджа, опередивший свое время, содержал все основные компоненты вычислительных машин, появившихся почти 100 лет спустя. Основная его идея не была забыта, она сыграла важную роль в дальнейшем развитии средств обработки информации. Эта идея полностью исключала участие человека в вычислительном процессе, сводя его роль к подготовке необходимых числовых данных и, как и в случае с ткацким станком Жаккарда, составлению программы, то есть плана выполнения вычислений, зафиксированного в некоторой специальной форме. Собственно процесс обработки информации должен был выполняться автоматически по заданной программе. Несмотря на то, что аналитическая машина Бэббиджа имелась только в виде проекта, для нее была составлена первая в мире программа. В 1843 году Ада Лавлейс, дочь английского поэта Джорджа Байрона, опубликовала работу, в которой были заложены основы современного программирования. Ею же была составлена программа вычисления так называемых чисел Фибоначчи.

\subsection{Электромеханическая обработка информации} \label{subsect1_1_5}

В связи с появлением электрических устройств и началом развития электротехники в конце XIX века начался следующий, электромеханический этап в развитии средств обработки информации. Отличительной чертой этого этапа является сочетание при выполнении вычислительных операций механических перемещений с работой электрических устройств. Первым такого рода устройством считается табулятор – машина, автоматизирующая выполнение простых вычислений на основе данных, нанесенных в виде пробивок на перфокарты. При этом какие-либо программы вычислений в табуляторах не использовались, а вычислительные операции, как правило, сводились к считыванию с перфокарт больших массивов числовых данных и их последующему суммированию. Первый табулятор был создан Германом Холлеритом в 1887 году. Основу этого устройства составляли простейшие электромеханические реле. Табуляторы широко использовались для выполнения расчетов статистического характера, например для проведения переписи населения в конце XIX века в США, Канаде, России и некоторых других странах. Для производства табуляторов Г. Холлерит в 1897 году организовал фирму Tabulating Machine Company, которая впоследствии преобразовалась в фирму IBM (International Business Machines corporation) – признанного и широко известного в настоящее время мирового лидера в сфере компьютерного производства. Различного рода табуляторы весьма эффективно использовались во всем мире для самых разных расчетов (статистических, астрономических, экономических и т. д.) вплоть до середины XX века.

В тридцатых годах XX века в разных странах начались разработки принципиально иных устройств – программно-управляемых релейных вычислительных машин. Считается, что первая в мире программно-управляемая электромеханическая вычислительная машина под названием «Z-3» была создана Конрадом Цузе в Германии в 1939-1941 годах. Эта машина могла «помнить» до 64 чисел одновременно и выполняла сложение двух чисел за 0,3 с (секунды), а умножение – за 5 с. Однако возможности и этой, и созданной после войны более совершенной модели «Z-4» по составлению программ были достаточно скромными. В частности, не было возможности осуществлять программный выбор одного из нескольких возможных вариантов действий. Это не позволяет считать «Z-3» универсальной вычислительной машиной.
Полностью идеи Чарльза Бэббиджа впервые были реализованы в машине «Марк-1», разработанной в фирме IBM под руководством Говарда Айкена в 1937-1944 годах. Эта машина считается первой в мире универсальной программно-управляемой вычислительной машиной. Вместе с тем устройство для выполнения арифметических действий в машине «Марк-1» было чисто механическим. Затем, в 1947 году, была построена полностью релейная (то есть электромеханическая) машина «Марк-2». Она выполняла одну операцию умножения за 0,7 с \cite{bib103}.

\section{История информатики} \label{sect1_2}

\subsection{Информатика как гуманитарная наука} \label{subsect1_2_1}

Термин «Информатика» появился в научной литературе в начале 60-х годов XX–го века. При этом он стал практически одновременно использоваться как в России, так и во Франции. Однако, смысловое значение этого термина во Франции (а затем и в других странах Западной Европы) и в России было в те годы совершенно различным. Так, например, в изданном в 1982 году во Франции Словаре информатики она определялась как наука о содержательной обработке, особенно при помощи автоматических машин, информации, которая, в свою очередь, рассматривалась как основа человеческих знаний и коммуникаций в технологической, экономической и социальной областях \cite{bib202}.

В тоже время в Большой советской энциклопедии информатика определялась тогда как теория научной информации, т.е., «область гуманитарного знания, изучающая структуру и общие свойства научной информации, а также основные закономерности процессов информационной коммуникации» \cite{bib203}(348-350). 

Необходимо отметить, что эта точка зрения на предмет информатики, как гуманитарной научной дисциплины, была сформирована в России группой ученых Всесоюзного института научно-технической информации \cite{bib204,bib205}. Развитие информатики как информационной науки достаточно подробно рассматривается в работе Р.С. Гиляревского \cite{bib206}.  Однако даже тогда далеко не все российские ученые рассматривали информатику только как информационную науку. Одним из них был профессор А.В. Соколов, который еще в 1971 году опубликовал статью, где прогнозировал, что информатика в будущем должна превратиться в обобщающую научную дисциплину всего коммуникационного цикла  и стать новым научным направлением, которое будет изучать не только научно-техническую информацию, но и все другие виды социальной информации и социальной коммуникации \cite{bib207}.

       В 1976 году академик А.П. Ершов в своем предисловии к монографии «Информатика – вводный курс» ввел новый термин «Информационная технология», которым он обозначил всю сферу машинной обработки информации. При этом он отметил, что одой из важнейших задач информатики как науки является изучение информационных технологий. С тех пор этот термин прочно вошел в терминологический арсенал информатики и до сих пор является одним из наиболее употребительных.

\subsection{Информатика как техническая наука} \label{subsect1_2_2}

Следующий этап переосмысления содержания предмета информатики следует отнести к периоду 1978-1983 г.г. Он был связан со стремительным развитием вычислительной техники и электроники и началом их все более широкого распространения не только в научной и  оборонной сферах, но также в промышленности, экономике, в сфере административного управления и финансовой  деятельности. Знаковыми для развития информационной сферы деятельности в этот период явились три основных события: 
\begin{itemize}
\itemМеждународный конгресс по информатике в Японии, который состоялся в 1978 году и дал новое, более широкое определение содержания термина «Информатика»;
\item объявление Японией национального проекта создания и использования ЭВМ пятого поколения, которое было воспринято мировым сообществом как технологический вызов и в значительной степени способствовало изменению взглядов на роль информатики в дальнейшем развитии общества;
\item создание в 1983 году в Академии наук СССР Отделения информатики, вычислительной техники и информатизации, на которое было возложено научное руководство работами в этой сфере деятельности как в нашей стране, так и в странах СЭВ.
\end{itemize}
         В этот период развития информатики на первый план выдвинулись ее инструментально-технологические аспекты, в то время как философские и социально-экономические аспекты этой науки еще находились в начальной стадии разработки. Свидетельством этого может служить определение информатики, которое было выработано на указанном выше Международном конгрессе по информатике в Японии. Звучит оно следующим образом: «Понятие информатики охватывает области, связанные с разработкой, созданием, использованием и материально-техническим обслуживанием систем обработки информации, включая машины, оборудование, математическое обеспечение, организационные аспекты, а также комплекс промышленного, коммерческого, административного, социального и политического воздействия».

Из приведенного определения видно, что, несмотря на явное доминирование в нем инструментально-технологических аспектов, уже тогда, около 30 лет тому назад, все же указывалось также и на социально-экономические, и на политические аспекты развития информатики. И, как показала история, этот прогноз оказался правильным \cite{bib201}.

\subsection{Современные представления о предмете информатики} \label{subsect1_2_3}

Мировые научные и образовательные  сообщества в настоящее время имеют три основных точки зрения на область исследований и предмет информатики. Согласно первой точке зрения, информатика --- комплексная техническая дисциплина, изучающая средства и методы автоматизированной обработки и передачи информации с использованием современных средств информатизации: ЭВМ и телекоммуникационные сети.

В нашей стране эта точка зрения сформировалась в значительной степени благодаря позиции таких авторитетных ученых, как академики А.А. Дородницин, А.А. Самарский, Н.Н. Моисеев, В.И. Сифоров, В.М. Глушков. Именно эта точка зрения была доминирующей вплоть до 1995 года и определяла отношение к информатике, как в отечественной науке, так и в системе образования.

Что же касается зарубежной науки, то в США, Канаде и многих других англоязычных странах русскоязычному термину «Информатика» сегодня соответствует, как минимум, четыре англоязычных термина и четыре сформировавшиеся области знания: Information science, Computer science, Computer Science and Information Science и Computational Science \cite{bib301}. При этом в «Computer science», само содержание которого подчеркивает компьютерную ориентацию данной области научных исследований и прикладных разработок, основное внимание уделяется инструментально-техническим аспектам, а не  изучению собственно информационных процессов, которыми занимается другая наука, получившая в этих странах название «Information science». В России аналогичная наука также сформировалась как целостное самостоятельное научное направление \cite{bib204,bib205,bib206}. 

Существует также точка зрения, в рамках которой информатика рассматривается одновременно и как фундаментальная естественная наука, и как комплексная область практической деятельности. Эта точка зрения стала все больше распространяться в России, начиная с 1990 года, когда были опубликованы  результаты исследований развития определений информатики  и ее предметной области, проведенные в Институте проблем информатики РАН \cite{bib208,bib209}. В этих работах была проанализирована  эволюция представлений о предмете информатики и показано ее место в системе науки,   междисциплинарное значение и взаимосвязи с другими научными дисциплинами. 

Таким образом, предметная область информатики так, как ее сегодня понимают различные российские ученые, гораздо многозначнее и шире, чем предметная область той дисциплины, которую в странах Запада принято обозначать термином «Computer science». Иначе говоря, термин «Информатика» обозначает сегодня предметную область, которая включает одновременно проблематику и «Computer science», и «Information science», но не только эти науки.


%============================================================================================================================

\clearpage