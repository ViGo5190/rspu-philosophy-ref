\chapter{История развития информатики} \label{chapt1}

Развитие вычислительной техники в XX веке стало основополагающим в  формировании информатики как науки.

Понятие информатики возникло  во Франции в 60-х гг. Так решили назвать область знаний, изучающую применение электронных вычислительных машин для автоматизации обработки информации. Слово информатика образовано путем слияния французских слов информация и автоматика. В англоязычных странах вместо «информатики» часто используют термин «computer science» (компьютерная наука).

В 40-х годах XX века ещё не существовало "информатики" - была лишь "компьютерная наука" ( хотя сейчас эти понятия и принято считать тождественными ), представляющая собой объединение возможностей электронно-вычислительных машин, математической логики и теории алгоритмов. Усовершенствования ЭВМ позволило в компьютерной науке появится новым направлениям.  \cite{bib102}

sdvs\cite{bib201}


\section{Предыстория информатики} \label{sect1_1}

\textbf{Предыстория} информатики уходит корнями в историю человеческого общества. Предысторию информатики можно разбить на несколько этапов, каждый из которых характеризуется резким изменениям в возможностях передачи, хранения и обработки информации.

\subsection{Начальный этап предыстории информатики} \label{subsect1_1_1}

Освоение человеком развитой устной речи является начальным этапом предыстории информатики. Членораздельная речь и язык стали основанием для формирования специфических социальных средств хранения и передачи информации. 

\subsection{Возникновение письменности} \label{subsect1_1_2}

Вторым этапом предыстории информатики является возникновение письменности. Этот этап ярко выражен резким ростом возможностей хранения информации.  Человек получил искусственную внешнюю память. В качестве средств передачи информации человек стал использовать различные почтовые службы. Начало развития наук стало возможным благодаря появлению письменности. На этом же этапе и появилось понятие "натуральное число". Все народы, обладавшие письменностью, владели понятием числа и пользовались той или иной системой счисления.

\subsection{Возникновение письменности} \label{subsect1_1_3}




\section{История информатики} \label{sect1_2}

Тут что-то из библиографии \cite{bib102}. 



%\newpage
%============================================================================================================================

\clearpage