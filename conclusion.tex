\chapter*{Заключение}						% Заголовок
\addcontentsline{toc}{chapter}{Заключение}	% Добавляем его в оглавление


Представленные и рассмотренные в данной работе исследования истории информатики как науки позволяют сделать следующие выводы:
\begin{itemize}

\item Информатика, в результате ряда эволюционных процессов, стала наукой не только о информационных процессах и технологиях в технических системах, но также и о закономерностях и методах реализации информационных процессов в природе и обществе.
\item  В настоящее время информатика должна квалифицироваться как такая же самостоятельная отрасль 
науки, как математика, физика, химия, биология и другие фундаментальные науки. Но не стоит и забывать и о междисциплинарном характере информатики, который имеет первостепенное значение для дальнейшего развития всего естествознания, а также гуманитарных и социальных наук.
\item Информатика, стремительно расширяющая свою предметную область, выполняя в тоже время интеграционные функции в системе научного познания аналогично тому, как это  было в период становления кибернетики во второй половине 20-го века, является одной из наиболее перспективных "точек роста" фундаментальной науки. При этом исключительно актуальной проблемой в настоящее время является развитие философских основ информатики, которое должно происходить на основе формирования новых философских представлений о природе информации. Формированию этих представлений в значительной степени содействуют те  результаты, которые  были получены в России, как в середине 60-х годов прошлого века, так и в самые последние годы, в том числе, в области развития информационных аспектов теоретической физики, синергетики, генетики, общей физиологии и ряда других наук.
\item  Несмотря на использование различных терминов, ученые всего мира изучают одну науку --- науку о передачи, обработке и хранении информации.
\end{itemize}

Именно информатика даёт человечеству надежду на то, что в обозримом будущем будут созданы принципиально новые эффективные технологии, которые позволят решить многие глобальные проблемы развития современного общества. Процесс формирования на нашей планете принципиально нового сообщества людей --- информационного общества, стал возможен лишь благодаря научной базе, что дает информатики. 

\clearpage