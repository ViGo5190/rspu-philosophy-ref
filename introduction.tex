\chapter*{Введение}							% Заголовок
\addcontentsline{toc}{chapter}{Введение}	% Добавляем его в оглавление


Современный мир невозможно представить без средств автоматизированной обработки, передачи и хранения информации. Повсеместное распространения ЭВМ привело к трансформации жизни всего человечества. И всё это стало возможно лишь благодаря молодой науке информатике.

Информатика как наука стала развиваться с середины прошлого столетия, что связано с появлением ЭВМ и начавшейся компьютерной революцией. Появление вычислительных машин в 1950-е гг. создало для информатики необходимую аппаратную поддержку, т.е. благоприятную среду для ее развития как науки. Всю историю информатики принято подразделять на два больших этапа: предысторию и историю.
 сравнению с предыдущим этапом, возможностей хранения, передачи и обработки информации.

\textbf{Актуальность} темы состоит в том, что за последние 70 лет ЭВМ вошли  в повседневную жизнь человечества. Темпы роста компьютеризированности человечества лишь возрастают. Информация стала главным ресурсом, а обработка, передача и хранение информации стало основной деятельностью человечества.  

\textbf{Целью} данной работы является рассмотреть основные этапы истории развития информатики, взгляды различных ученых на то, что же скрывается за словом "информатика".

\textbf{Задачами} данной работы является:
\begin{itemize}
\item Изучить историю и предысторию информатики;
\item Сравнить взгляды ученых разных стран на определение "компьютерной науки";
\item Рассмотреть перспективные направления развития информатики.
\end{itemize}

\clearpage