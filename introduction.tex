\chapter*{Введение}							% Заголовок
\addcontentsline{toc}{chapter}{Введение}	% Добавляем его в оглавление


Современный мир невозможно представить без средств автоматизированной обработки, передачи и храниния информации. Повсеместное распространения ЭВМ привело к транформации жизни всего человечества. И всё это стало возможно лишь благодаря молодой науке информатике.

Информатика как наука стала развиваться с середины прошлого столетия, что связано с появлением ЭВМ и начавшейся компьютерной революцией. Появление вычислительных машин в 1950-е гг. создало для информатики необходимую аппаратную поддержку, т.е. благоприятную среду для ее развития как науки. Всю историю информатики принято подразделять на два больших этапа: предысторию и историю.
 сравнению с предыдущим этапом, возможностей хранения, передачи и обработки информации.

\textbf{Актуальностью} данной работы является \ldots

\textbf{Целью} данной работы является \ldots

\textbf{Задачи} данной работы является \ldots

\clearpage