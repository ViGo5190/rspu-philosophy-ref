\chapter{Как будет развиваться} \label{chapt2}



Еще одна точка зрения, которая в последние годы приобретает все больше сторонников, как в России, так и за рубежом, связана с работами российских ученых А.П. Ершова, Ю.И. Шемакина, Ю.А. Шрейдера и А.Д. Урсула, которые еще 20 лет назад рассматривали информатику как формирующуюся новую фундаментальную науку, которая будет иметь первостепенное значение не только для всего естествознания, но и для гуманитарных наук. Близкую точку зрения высказал в 1963 году американский ученый С. Горн. Ее изложение можно найти в работе \cite{bib301}.

Этот прогноз российских и американских ученых был основан на признании фундаментальности понятия информации, которая собственно и является важнейшим объектом изучения  информатики как фундаментальной науки, востребованности возможностей информатики разными областями знания, а также на гипотезе, согласно которой информационные закономерности должны иметь общую основу для своего проявления, как в живой, так и в неживой природе, в том числе – и в искусственно созданных человеком технических системах. Автор данной работы является сторонником именно этой точки зрения.

\section{Одиночное изображение} \label{sect2_1}



%\newpage
%============================================================================================================================
\section{Длинное название параграфа, в котором мы узнаём как сделать две картинки с общим номером и названием} \label{sect2_2}



%\newpage
%============================================================================================================================
\section{Пример вёрстки списоков} \label{sect2_3}




\clearpage