\chapter{Направления развития информатики} \label{chapt2}

\section{Комплексный характер проблем информатики} \label{sect2_1}

Необходимость комплексного исследования проблем информатики не только в рамках академического сектора науки, но также и в системе образования, достаточно подробно рассмотрена в работах \cite{bib302,bib210,bib211,bib303}. Главный вывод, который можно сделать из анализа содержания этих работ, заключается в том, что наступает новый период развития информатики как междисциплинарного научного направления, которое будет выполнять интеграционные функции для других направлений науки, как естественнонаучных, так и гуманитарных. Проникновение идей и методов информатики в эти области диктуется сегодня потребностями и логикой развития самой фундаментальной науки, а также необходимостью решения ряда важных прикладных проблем. Следует ожидать, что это проникновение не только даст новый импульс для развития научных исследований на стыке информатики с другими науками, но также обогатит и саму информатику новыми перспективными идеями \cite{bib105}.

Указанная выше тенденция стала особенно заметно проявлять себя в России в последние годы, когда в научной печати был опубликован целый ряд статей и монографий, содержание которых свидетельствует о том, что идеи и методы фундаментальной информатики находят сегодня все большее распространение в теории систем, синергетике, общей физике, квантовой механике, теоретической биологии, физиологии, генетике, социологии и других научных дисциплинах \cite{bib105}.

Многообразие подходов к определению предмета и основных задач информатики как науки в настоящее время является вполне закономерным. В значительной степени оно обусловлено многообразием современных представлений об информации, которая является фундаментальным понятием современной науки, но до сих пор еще не имеет однозначного определения. Причина этого заключается в том, что феномен информации по- разному проявляет себя в различных информационных средах, т.е. в тех конкретных условиях, в которых протекают информационные процессы, закономерности и методы реализации которых и изучает информатика как фундаментальная наука. Поэтому в различных направлениях развития информатики (техническом, биологическом, социальном, физическом) анализируются лишь вполне определенные аспекты проявления феномена информации и информационных процессов, которые обусловлены тем или иным видом информационной среды \cite{bib201,bib212}.

Таким образом, именно фундаментальность понятия информации и ключевая роль информационных процессов в развитии живой и неживой природы и являются теми основными факторами, которые выдвигают информатику на уровень фундаментальных наук и ставят ее в один ряд с такими науками, как общая теория систем, синергетика, кибернетика, физика, химия, биология \cite{bib105}.


\section{Междисциплинарное взаимодействие} \label{sect2_2}

Развитие информатики в ближайшее десятилетие будет проходить в тесном взаимодействии с другими научными дисциплинами, что в дальнейшем сможет квалифицироваться как самостоятельная отрасль науки, имеющая такое же значение, как физика, химия, биология, психология и другие. Информатика, сочетающая в себе как естественнонаучные, так и гуманитарные аспекты, при междисциплинарных исследованиях  будет порождать новые научные результаты.

\subsection{Философские основы информатики} \label{sect2_2_1}

Формирование философии информации, как одного из самостоятельных направлений развития философии --- направление, основоположником которого является российский философ А.Д. Урсул\cite{bib213}, сегодня развивают некоторые специалисты из Китая, Великобритании и США. 

Второе направление заключается в изучении философских проблем информатики и ее роли в развитии других наук и формировании нового научного мировоззрения \cite{bib212}. Можно предположить, что в дальнейшем произойдет интеграция этих двух направлений и на их основе будет сформирована новая научная дисциплина – метаинформатика. Эта дисциплина должна будет вобрать в себя наиболее общие концептуальные результаты указанных выше двух направлений и стать основой для формирования новой научной парадигмы, основанной на более тесной интеграции естественного и гуманитарного направлений в науке с целью получения целостного знания о природе, человеке и обществе.



\subsection{Развитие технической информатики} \label{sect2_2_2}

В настоящее время возможности микроэлектроники еще не исчерпаны, но давление пределов уже ощутимо. Внедрение новых технологий производства микропроцессоров испытывает экономические проблемы. Например, построение нового завода по производству микросхем с 0,13-микронной технологией стоит от 2 до 4 млрд. долларов. Это заставляет искать альтернативные пути построения будущих ЭВМ. Интенсивные поиски идут по многим направлениям. Наиболее перспективные из них:
\begin{itemize}
\item создание молекулярных и биокомпьютеров (нейрокомпьютеров);
\item разработка квантовых компьютеров;
\item разработка оптических компьютеров.
\end{itemize}

Качественно новые и социально значимые результаты следует ожидать и от развития промышленного производства гибких биологических экранов отображения информации. Их применение в сфере образования позволит использовать новые педагогические технологии, ориентированные на более активную работу правого полушария головного мозга человека, ответственного за пространственное воображение и образное мышление\cite{bib105}.


\subsection{Развитие социальной информатики} \label{sect2_2_3}

Здесь уже сегодня разворачиваются исследования в области информационной культурологи, информационной антропологии и создания информационной концепции искусства и творчества. Хуже обстоит дело с разработкой научной теории развития информационного общества, основанного на знаниях, теории информационных ресурсов, информационной экономики и гуманитарных проблем информационной безопасности, включая проблемы виртуализации общества.

\subsection{Развитие биоинформатики} \label{sect2_2_3}

На рубеже 20-21 веков биоинформатика превратилась в бурно развивающуюся область мировой биомедицинской науки. Наряду с исследователями, ведущими фундаментальные разработки, потребителями биоинформационных технологий являются медицинские, фармакологические, биотехнологичные и учебные учреждения. Эта область науки определена в качестве приоритетной как в США, так и во всех других развитых странах. 

Количество центров биоинформатики постоянно растет во всех странах Европы, Азии, США и Австралии. Наряду с государственными, академическими и образовательными центрами биоинформатики, в последние годы возникло значительное число организаций и проектов, ориентированных на коммерческое использование результатов исследований в области биоинформатики. Это прежде всего организации, деятельность которых ориентирована на структурный, функциональный и сравнительный анализ геномов, включая геном человека. Наряду с применением уже созданных методов биоинформатики интенсивно развивается техническая и программная база для решения прикладных задач, особенно в фармакологии. Быстрыми 
темпами совершенствуется также и индустрия программного обеспечения для решения таких задач\cite{bib214}.

\subsection{Информатика в системе образования} \label{sect2_2_4}


Практически во всех современных энциклопедических словарях информатика определяется как комплексное междисциплинарное научное направление. Она оказывает большое влияние на многие другие области научных исследований, передавая им свою научную методологию, главными достижения которой сегодня следует считать методологию информационного моделирования, а также информационный подход к анализу различных объектов, процессов и явлений в природе и обществе. Именно поэтому изучение информатики как фундаментальной науки в системе образования имеет исключительно большое значение для формирования современного научного мировоззрения.

К сожалению, следует констатировать, что такой подход к изучению информатики, хотя и продекларирован в документах ЮНЕСКО, в трудах Российской академии наук, а также в проектах новых государственных образовательных стандартов России, практически еще очень медленно внедряется в систему образования. Причина здесь не только в отсутствии хороших учебников по информатике для высшей и средней школы, но, главным образом, в консервативности мышления чиновников, работающих в сфере образования России, которые и сегодня все еще продолжают считать информатику вспомогательной технической дисциплиной о компьютерных технологиях и телекоммуникационных системах.

В настоящее время опубликовано значительное количество работ, где отмечается, что осмысление определяющей роли информации в эволюционных процессах природы и общества открывает совершенно новую, информационную Картину Мира, которая существенным образом отличается от традиционной вещественно-энергетической Картины мироздания, доминировавший в науке еще со времен Декарта и Ньютона до конца XX–го века. Можно ожидать, что формирование такой новой Картины Мира будет осуществлено в науке в ближайшие  десятилетия и это должно стать основой формирования новой научной парадигмы, в которой информационным аспектам будет отведена существенно более важная роль по сравнению с тем, как это имеет место в настоящее время. Эта парадигма должна привести и к формированию новой парадигмы самой информатики, философские основы которой, конечно же, должны изучаться и в системе образования \cite{bib304}.









\clearpage